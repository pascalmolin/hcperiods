\documentclass[main.tex]{subfiles}

\renewcommand\d{\mathrm{d}\,}
\newcommand\set[1]{\left\{#1\right\}}
\newcommand\abs[1]{\left|#1\right|}
\newcommand\fab{\tilde{f_{a,b}}}
\DeclareUnicodeCharacter{3B1}{\alpha}
\DeclareUnicodeCharacter{3BB}{\lambda}
\DeclareUnicodeCharacter{3C4}{\tau}
\DeclareUnicodeCharacter{3C0}{\pi}
\DeclareUnicodeCharacter{3C6}{\varphi}
\begin{document}

  \section{Double-exponential integration}

We want to evaluate
\begin{equation}
    \int_a^b\omega_{i,j} = C_{i,j}\int_\R g(t)\dt
\end{equation}
where
\begin{equation}
    g(t) = \frac{u(t)^{i-1}}{\fab(u(t))^j}\frac{λ\cosh(t)}{\cosh(λ\sinh(t))^{2α}}
\end{equation}
with $α=(1-j/N)$.

Let $Z_\tau = \set{\tanh(λ\sinh(z), \Im(z)<\tau }$ be the image
of the strip $\Delta_\tau$ under the change of variable $u=\tanh(λ\sinh(t))$.

Since we can compute the distance of each branch point $u_i$ to
both $[-1,1]$ and its neighborhood $Z_\tau$, we obtain
  \begin{lemma}
      There exist explicitly computable
      constants $M_1$, $M_2$ such
      that
      \begin{itemize}
          \item for $u\in[-1,1]$, $\abs{\frac{u^{i-1}}{\fab(u)^{j}}}\leq M_1$
          \item for $u\in Z_\tau$, $\abs{\frac{u^{i-1}}{\fab(u)^{j}}}\leq M_2$.
      \end{itemize}
  \end{lemma}


  \begin{lemma}[truncation error]
      \begin{equation}
          \sum_{\abs{k}>n}\abs{hg(kh)}
          \leq \frac{2^{2α} M_1}{αλ}\exp(-2αλ\sinh(nh))
      \end{equation}
  \end{lemma}
  \begin{proof}
      We bound the sum by the integral of a decreasing function
      \begin{align*}
          \sum_{\abs{k}>n}\abs{hg(kh)}
          &\leq2M_1\int_{nh}^\infty\frac{λ\cosh(t)}{\cosh(λ\sinh(t))^{2α}}
          =2M_1\int_{\sinh(nh)}^\infty\frac{\dt}{\cosh(λ t)^{2α}}\\
          &\leq 2^{2α+1} M_1\int_{\sinh(nh)}^\infty e^{-2αλ t}\dt
          = \frac{2^{2α} M_1}{αλ}e^{-2αλ\sinh(nh)}
      \end{align*}
  \end{proof}

  \begin{lemma}[discretization error]
      \begin{equation}
          \sum_{k\neq0}\abs{\hat g(\frac kh)}
          \leq
          \frac{M_2B(τ,α)}{e^{2πτ/h}-1}
      \end{equation}
      where
      \begin{equation}
          B(τ,α)
          =
          \frac{2}{\cosτ}
          \left(
      \frac{X_1}2(\frac1{\cos^{2α}(λ\sinτ)}+\frac1{X_1^{2α}})
      +\frac{1}{2α\sinh^{2α}X_1}
      \right)
      \end{equation}
      with
      $X_1=\cos(τ)\sqrt{\frac{π}{2λ\sinτ}-1}$.
  \end{lemma}


  The proof follows the same lines as the one in \cite{PMthese}.

  We first bound the Fourier transform by a shift of contour
  \begin{equation}
      \forall X>0, \hat g(\pm X) = e^{-2πXτ} \int_{\R} g(t\mp iτ) e^{-2iπtX}\dt
  \end{equation}
  so that introducing $φ(t)=λ\sinh(t+iτ)$ we have
  \begin{equation}
      \sum_k \abs{\hat g(\frac kh)}
      \leq
      \frac{2M_2}{e^{2πτ/h}-1}\int_\R \abs{
      \frac{λ\cosh(t+iτ)}{\cosh(λ\sinh(t+iτ))^{2α}}}\dt
  \end{equation}

  Now $φ(t) = X(t)+iY(t)$ lies on the hyperbola
  $Y^2 =λ^2(\sin^2τ+\tan^2 τX^2)$, and
  \begin{align}
      \abs{λ\cosh(t+iτ)} &\leq λ\cosh(t) =\frac{X'(t)}{\cos(τ)}\\
      \abs{\cosh(X+iY)}^2 &= \sinh(X)^2+\cos(Y)^2 
  \end{align}
  so that
  \begin{equation}
      \int_\R \abs{
      \frac{λ\cosh(t+iτ)}{\cosh(λ\sinh(t+iτ))^{2α}}}\dt
      \leq
      \frac{2}{\cosτ}\int_0^\infty\frac{\d X}{(\sinh(X)^2+\cos(Y)^2)^\alpha}
  \end{equation}
  For $X_0=0$, $Y_0=λ\sinτ<\frac{π}2$, and $Y_1=\frac{π}2$ for
  $X_1=\cos(τ)\sqrt{\frac{π}{2Y_0}-1}$.

  We cut the integral at $X=X_1$ and write
  \begin{align}
      \int_0^{X_1}\frac{\d X}{(\sinh(X)^2+\cos(Y)^2)^\alpha}
      & \leq \int_0^{X_1}\frac{\d X}{(X^2+\cos^2Y)^α} \\
      \int_{X_1}^\infty\frac{\d X}{(\sinh(X)^2+\cos(Y)^2)^\alpha}
      & \leq \int_{X_1}^\infty\frac{\d X}{(\sinh X)^{2α}}
  \end{align}
  
  We bound the first integral by convexity:
  since $Y(X)$ is convex and $\cos$ is concave decreasing for $Y\leq Y_1$ we
  obtain by concavity of the composition
  \begin{equation}
      \forall X\leq X_1, \cos(Y)\geq \cos(Y_0)(1-\frac{X}{X_1})
  \end{equation}
  Now $X^2+\cos^2Y\geq P_2(X)$ where
  \begin{equation}
     P_2(X) = X^2(1+\frac{\cos^2(Y_0)}{X_1^2})-2\frac{\cos^2(Y_0)}{X_1}X+\cos^2(Y_0)
  \end{equation}
  is a convex quadratic, so $X\mapsto P_2(X)^{-α}$ is still convex and the integral
  is bounded by one trapeze
  \begin{equation}
      \int_0^{X_1}\frac{\d X}{P_2(X)^α}\leq X_1\frac{P_2(0)+P_2(X_1)}2
      = \frac{X_1}2\left(\frac1{\cos^{2α}(Y_0)}+\frac1{X_1^{2α}}\right)
  \end{equation}

  For the second integral we use
  $\sinh(X)\geq\sinh(X_1)e^{X-X_1}$ to obtain
  \begin{equation}
      \int_{X_1}^\infty \frac{\d X}{\sinh(X)^{2α}} \leq \frac1{2α\sinh(X_1)^{2α}}
  \end{equation}

\iffalse
  \begin{align}
      \sinh(x+iy) &= \sinh x\cos y+i\cosh x\sin y\\
      \cosh(x+iy) &= \cosh x\cos y+i\sinh x\sin y
  \end{align}
  so that writing $λ\sinh(t+iτ)=X+iY$ we express the integral in terms
  of $X,Y$ with
  \begin{align}
      X &= λ\sinh t\cosτ\\
      Y &= λ\cosh t\sinτ\\
      Y^2 =λ^2(\sin^2τ+\tan^2 τX^2)\\
      λ\cosh(t+iτ) &= λ\cosh t\cos τ+iλ\sinh t\sinτ \\
                     &= Y/\tan(τ) + i X\tan(τ)\\
      \abs{\cosh(X+iY)}^2
      &= \cosh^2 X\cos^2 Y+\sinh^2 X\sin^2 Y \label{eq:boundchcos}\\
      &= \sinh^2 X + \cos^2 Y \label{eq:boundchsh}
  \end{align}
\fi
      
\end{document}
