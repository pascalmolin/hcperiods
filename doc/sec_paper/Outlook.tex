\documentclass[main.tex]{subfiles}

\begin{document}

  \section{Outlook}\label{sec:outlook}

  In this paper we presented an approach based on numerical integration for
  multiprecision computation of period matrices and the Abel-Jacobi map of
  superelliptic curves given by $m > 1$ and squarefree $f \in \C[x]$.
 
  Integration along a spanning tree and the special geometry of such curves
  make it possible to compute these objects to high precision performing only
  a few numerical integrations. The resulting algorithm has an excellent
  scaling with the genus and works for several thousand digits of precision.

  \subsection{Reduced small period matrix}

   For a given curve our algorithm computes a small period matrix
   $\tau$ in the Siegel upper half-space $\mathcal{H}_g$ which is arbitrary
   in the sense that it depends on the choice of a symplectic basis made
   during the algorithm.
   
   For applications like the computation of theta functions it is useful to
   have a small period matrix in the Siegel fundamental domain $\mathcal{F}_g \subset
   \mathcal{H}_g$ (see \cite[\S 1.3]{PlaneQuarticsCM}).
  
   We did not implement any such reduction.
   The authors of \cite{PlaneQuarticsCM} give a theoretical sketch of
   an algorithm (Algorithm 1.9) that achieves this reduction step, as well as
   two practical versions (Algorithms 1.12 and 1.14) which work in any genus and have been implemented for $g
   \le 3$. It would be interesting to combine this with our implementation.
  
  \subsection{Generalizations}
  
  We remark that there is no theoretical obstruction to generalizing our
  approach to more general curves.
  
  \subsubsection{Multiple roots}
  \label{subsec:nonseparable}
  
  In a first step the algorithm could be
  extended to all complex superelliptic curves given by $m > 1$ and $f \in
  \C[x]$, where $f$ can have multiple roots of order at most $m-1$, say 
  $f = \prod_{k=1}^n (x-x_k)^{n_k}$.
  We want to highlight the following issues:
  \begin{itemize}
   \item[$\bullet$] The differentials are of the form
   $\frac{\prod_{k=1}^n (x-x_k)^{i_k}}{y^j}\dx$
    where the exact condition on the holomorphicity is given in \cite[Theorem 3]{Koo1991}.
   However, these holomorphic differentials can still be integrated using double-exponential integration as presented
   in \S \ref{m-subsec:de_int}.
   \item[$\bullet$] The local monodromies may no longer be equal or even cyclic, but they are completely (up
   to conjugacy) determined
   by the multiplicities $n_k$. We believe that applying the Tretkoff algorithm \cite{TT1984}
   to obtain a homology basis and the intersections could be a better approach than generalizing the methods
   used in Section \ref{m-sec:intersections}, although this seems possible.
  \end{itemize}

  Although several adjustments would have to be made in the analysis and in the
  code, staying within the superelliptic setting promises
  a fast and rigorous extension of our algorithm. 

  Moreover, this generalization would allow to perform any Moebius transform on the
  model of the curve and to efficiently implement the idea of \S \ref{m-subsec:improving}.

  \subsubsection{General affine algebraic curves}
  
  We also believe that the strategy employed here (numerical integration between
  branch points combined with information about local intersections) could
  be adapted to completely general algebraic curves given by $F \in \C[x,y]$.

  However, serious issues have to be overcome:
  \begin{itemize}
      \item On the numerical side we no longer have a nice $m$-th root function,
          it may be replaced by a combination of Newton's method between branch
          points (analytic continuation has to
          be performed on all sheets) and
          Puiseux series expansion around them.
      \item On the geometric side the combinatorics of loops and intersections
          become even more intricate than in the non-separable case \ref{m-subsec:nonseparable}.
          It is not clear whether our strategy
          based on shifting numbers and local intersection could be generalized.
          One can instead obtain the local monodromies from analytic continuation  
          and then employ the Tretkoff algorithm \cite{TT1984}, as described (for example) in \cite{FrauendienerKlein2016}.
  \end{itemize}
  We did not investigate further: at this point the advantages
  of superelliptic curves which are utilized by our approach are already lost
  (simple geometry of branch points and $m-1$ integrals at the cost of one).
  It is not clear whether this approach would
  be more efficient than methods that avoid branch points.

  \biblio
  \end{document}
