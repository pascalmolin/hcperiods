\documentclass[main.tex]{subfiles}

\begin{document}

  \newpage

  \section{Outlook}\label{sec:outlook}

  In this paper we presented an approach based on numerical integration for multiprecision computation of period matrices and the Abel-Jacobi map of superelliptic curves given by
 $m > 1$ and squarefree $f \in \C[x]$.
 
  Integration along a spanning tree and the special geometry of such curves make it possible to compute these objects too high precision performing only a few numerical integrations.
  The resulting algorithm has an excellent scaling with the genus and works for several thousand digits of precision.
  
  \subsection{Generalizations}
  
  We remark that there is no theoretic obstruction to generalizing this approach to more general curves.
  In a first step the algorithm could be extended to all complex superelliptic curves given by $m > 1$ and $f \in \C[x]$, where $f$ can have multiple roots of order at most $m-1$.
  Although several adjustments would have to be made (e.g.\ Differentials, Homology, Integration), staying within the superelliptic setting promises a fast and rigorous extension of our algorithm. 
  
  \medskip
  
  For completely general algebraic curves given by $f \in \C[x,y]$ it is very hard to make this approach work.
  It is unclear how to integrate between branch points numerically without using Puiseux series expansions (and splitting up each integral into three parts).  
  All advantages of superelliptic curves that are utilized by our approach are lost.
  Even though a generalization is possible, we highly doubt that this is the most efficient approach.
  
  \subsection{Reduced small period matrix}

   For a given curve our algorithm computes an arbitrary small period matrix $\tau$ in the Siegel upper half-space $\mathcal{H}_g$ in the sense that we can't say
   anything about its imaginary part. 
   
   For applications like the computation of theta functions it is useful
   to have a small period matrix in the fundamental domain $\mathcal{F}_g \subset \mathcal{H}_g$ (see \cite[\S 1.3]{PlaneQuarticsCM}).
  
   Moreover, the authors of \cite{PlaneQuarticsCM} give a theoretical sketch of an algorithm (Algorithm 1.9) that achieves this reduction step, as well as two practical versions (Algorithms 1.12 and 1.14) which at least
   work for $g \le 3$.

  \biblio
  \end{document}
